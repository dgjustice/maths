\documentclass{article}
\usepackage{amsmath}
\usepackage{amsfonts}

\begin{document}

\section{3.4.24}

Write up a formal proof that the p-harmonic series

\begin{equation}
	\sum_{k=1}^{\infty} \frac{1}{k^p}
\end{equation}

converges for p $>$ 1 using the method sketched in the text.

\paragraph{}

By definition, p is greater than one and thus positive.
1 divided by a positive number is a positive number.

\begin{equation}
	\forall k > 0, \frac{1}{k^p} > \frac {1}{(k+1)^p}
\end{equation}

The partial sums of the p-harmonic series are increasing, so the sequence and the series are monotonic.
Given (from the text pp. 143-144):

\begin{equation}
	\sum_{k=1}^{2n-1} \frac{1}{k^p} \le \sum_{k=1}^{\infty} \frac{2^{k-1}}{(2^{k-1})^p}
	= \sum_{j=0}^{\infty} (2^{1-p})^j = \frac{2^{p-1}}{(2^{p-1})-1}
\end{equation}

This formalizes the grouping of terms demonstrated in section 3.4.2.
The upper bound is the convergent geometric series.
Since the partial sums are increasing and bounded above, the p-harmonic series for $p > 1$ must be convergent.


\end{document}
