\documentclass[11pt]{article}
\usepackage[utf8]{inputenc} % Required for inputting international characters
\usepackage[T1]{fontenc} % Output font encoding for international characters
\usepackage{amsmath}
\usepackage{amsfonts}
\usepackage{pifont}
\usepackage[backend=biber,
style=apa,
citestyle=authoryear]{biblatex}

\addbibresource{references.bib}

\usepackage{mathpazo} % Palatino font

\begin{document}

%----------------------------------------------------------------------------------------
%	TITLE PAGE
%----------------------------------------------------------------------------------------

\begin{titlepage} % Suppresses displaying the page number on the title page and the subsequent page counts as page 1
	\newcommand{\HRule}{\rule{\linewidth}{0.5mm}} % Defines a new command for horizontal lines, change thickness here

	\center % Centre everything on the page

	%------------------------------------------------
	%	Headings
	%------------------------------------------------

	\textsc{\LARGE American Public University}\\[1.5cm] % Main heading such as the name of your university/college

	%------------------------------------------------
	%	Title
	%------------------------------------------------

	\HRule\\[0.4cm]

	{\huge\bfseries Assignment 1}\\[0.4cm] % Title of your document

	\HRule\\[1.5cm]

	%------------------------------------------------
	%	Author(s)
	%------------------------------------------------

	\begin{minipage}{0.4\textwidth}
		\begin{flushleft}
			\large
			\textit{Author}\\
			Daniel \textsc{Justice} % Your name
		\end{flushleft}
	\end{minipage}
	~
	\begin{minipage}{0.4\textwidth}
		\begin{flushright}
			\large
			\textit{Professor}\\
			Dr. Farshad \textsc{Foroozan} % Supervisor's name
		\end{flushright}
	\end{minipage}

	%------------------------------------------------
	%	Date
	%------------------------------------------------

	\vfill\vfill\vfill
	{\large\today} % Date, change the \today to a set date if you want to be precise
	\vfill

\end{titlepage}

%----------------------------------------------------------------------------------------

\section{}

\#1) Prove the following by using a contrapositive argument.
\paragraph{}
Let $q(x) = 6x^2+2$.
If $q(x_1)\ne q(x_2)$, then $x_1 \ne x_2$.
\paragraph{}
The contrapositive of $P \implies Q$ is $\neg Q \implies \neg P$.
The contrapositive of the previous statement is:
\begin{equation}
	x_1=x_2 \implies q(x_1) = q(x_2)
\end{equation}
$6x^2+2$ is a relation that maps $\{x, x \in \mathbb{R}\}$ to $\{y:y \ge 2, y \in \mathbb{R}\}.$
Every $x$ in the domain maps to exactly one $y$ in the codomain, so choosing any arbitrary $x_1 = x_2$ maps to $q(x_1)=q(x_2)$.
$\square$

\section{}

\#2) Clearly state the converse of the implication given in problem \#1.
Provide a counter example showing why the converse is false.
\paragraph{}
The previous problem can be stated as $P \implies Q$.
The converse of this statement is $Q \implies P$.
The converse of problem \#1 is:

\begin{equation}
	x_1 \ne x_2 \implies q(x_1) \ne q(x_2)
\end{equation}

This implication is quickly disproven by choosing any pair positive and negative values for a real number $\ne$ 0.
For example,

\begin{equation}
	\begin{aligned}
	2 \ne -2 \implies q(2) \ne q(-2) \\
	q(2) = 26 \\
	q(-2) = 26 \\
	\end{aligned}
\end{equation}

Since $2 \ne -2$ and $26 = 26$, the converse is clearly false.
$\square$

\section{}

\#3) Prove the following by using: a contradiction argument; the definition of a logarithm; the definition of rational numbers; and the properties of even and odd numbers.
\paragraph{}
The $log_8(27)$ is irrational.
\paragraph{}
According to Wolfram, "a logarithm $log_bx$ for a base $b$ and a number $x$ is defined to be the inverse function of taking $b$ to the power of $x$"(\cite{wolfram}).
This means that we are looking for some number $n$ where $8^n=27$.
As a contradiction, assume that $n$ is a rational number that can be represented as $\frac{p}{q}$ where $p$ and $q$ are coprime integers.
We can rewrite this as
\begin{equation}
	8^{\frac{p}{q}}=27
\end{equation}
This can be rewritten in terms of roots.
\begin{equation}
	\sqrt[q]{8^p}=27
\end{equation}
\begin{equation}
	8^p=27^q
\end{equation}
It is assumed that $p$ is an integer, so by the property of even numbers, $8^p$ must be even.
Furthermore, $q$ is also assumed to be an integer which means that $27^p$ is odd due to the property of odd numbers.
This leads to a contraction, thus $log_8(27)$ must be irrational.
$\square$

\section{}

\#4) Prove the following by using: a contrapositive argument; the fact that a sum of integers is an integer; the fact that a product of integers is an integer; and the definition of rational numbers.
\paragraph{}
If $xy$ is irrational, then $x$ is irrational or $y$ is irrational.
\paragraph{}
The contrapositive of $P \implies Q$ is $\neg Q \implies \neg P$.
The contrapositive of the previous statement is:
\begin{equation}
	x \in \mathbb{Q}, y \in \mathbb{Q} \implies xy \in \mathbb{Q}
\end{equation}
By the definition of rational numbers, let $x=\frac{p}{q}$ and $y=\frac{m}{n}$ where $p$, $q$, $m$, and $n$ are all integers(\cite{tbb}).
Rewriting $xy$ in these terms yields
\begin{equation}
	xy=\bigg(\frac{p}{q}\bigg)\bigg(\frac{m}{n}\bigg)=\frac{pm}{qn}
\end{equation}
The product of two integers is an integer, so $pm$ and $qn$ are both integers.
The product $xy$ has been rewritten in terms of an integer divided by an integer which is the definition of a rational number.
$\square$

\printbibliography

\end{document}
