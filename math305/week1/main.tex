\documentclass{article}
\usepackage{amsmath}

\begin{document}

\section{A.2.3(a)(c)}

\subsection{Part (a)}

Describe the sets

\begin{equation}
	\bigcup_{n=1}^{\infty}(-1/n,1/n)
\end{equation}

and

\begin{equation}
	\bigcap_{n=1}^{\infty}(-1/n,1/n)
\end{equation}

I'll start off with some concrete examples of the first set.
This is a series of open intervals of decreasing size.

\begin{equation}
	(-1,1) \cup (-\frac{1}{2},\frac{1}{2})\cup (-\frac{1}{3},\frac{1}{3})\cup (-\frac{1}{4},\frac{1}{4}) ...
\end{equation}

I see this as a bunch of nested boxes.
As $n$ increases, the interval (AKA the box) gets smaller.
Since this is a union and the bounds of the smaller box are narrower than the larger box, we can keep stuffing boxes into one another.
A bit of nuance here is that the endpoints of the smaller interval exist in the larger inverval.
The way I think of this is that we discard the wrapping paper of the smaller box when combining items.
I do not believe this results in any gaps (or holes) in the union, but feel free to correct my logic!
Due to the fact that we are working with sets, once we remove duplicates, we are left with the first and largest open inverval $(-1,1)$.

\paragraph{}

Here is an example of the second set.

\begin{equation}
	(-1,1) \cap (-\frac{1}{2},\frac{1}{2})\cap (-\frac{1}{3},\frac{1}{3})\cap (-\frac{1}{4},\frac{1}{4}) ...
\end{equation}

This results in the same progression of smaller intervals, but we are now dealing with the intersection of intervals.
We only account for items in both boxes and discard the excess.
As $n$ grows, the interval shrinks and so does the amount of common material between one box and the next.

\begin{equation}
	(-1,1) \cap (-\frac{1}{2},\frac{1}{2}) = (-\frac{1}{2},\frac{1}{2})
\end{equation}

\begin{equation}
	(-\frac{1}{2},\frac{1}{2}) \cap (-\frac{1}{3},\frac{1}{3}) = (-\frac{1}{3},\frac{1}{3})
\end{equation}

\begin{equation}
	(-\frac{1}{3},\frac{1}{3}) \cap (-\frac{1}{4},\frac{1}{4}) = (-\frac{1}{4},\frac{1}{4})
\end{equation}

The end result is an infinitely small {\bf open} interval centered around $0$.

\subsection{Part (c)}

Describe the sets

\begin{equation}
	\bigcup_{n=1}^{\infty}[-1/n,1/n]
\end{equation}

and

\begin{equation}
	\bigcap_{n=1}^{\infty}[-1/n,1/n]
\end{equation}

This is the same as example (3) except we are now dealing with a closed interval instead of an open one.

\begin{equation}
	[-1,1] \cup [\frac{-1}{2},\frac{1}{2}]\cup [\frac{-1}{3},\frac{1}{3}]\cup [\frac{-1}{4},\frac{1}{4}] ...
\end{equation}

Again, since we are combining nested boxes, the end result is the closed interval $[-1, 1]$.

\paragraph{}

Here is an example of the last set under consideration.

\begin{equation}
	[-1,1] \cap [\frac{-1}{2},\frac{1}{2}]\cap [\frac{-1}{3},\frac{1}{3}]\cap [\frac{-1}{4},\frac{1}{4}] ...
\end{equation}

\begin{equation}
	[-1,1] \cap [-\frac{1}{2},\frac{1}{2}] = [-\frac{1}{2},\frac{1}{2}]
\end{equation}

\begin{equation}
	[-\frac{1}{2},\frac{1}{2}] \cap [-\frac{1}{3},\frac{1}{3}] = [-\frac{1}{3},\frac{1}{3}]
\end{equation}

\begin{equation}
	[-\frac{1}{3},\frac{1}{3}] \cap [-\frac{1}{4},\frac{1}{4}] = [-\frac{1}{4},\frac{1}{4}]
\end{equation}

This results in an infinitely small {\bf closed} interval centered around $0$.

\end{document}
