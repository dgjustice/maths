\documentclass[11pt]{article}
\usepackage[utf8]{inputenc} % Required for inputting international characters
\usepackage[T1]{fontenc} % Output font encoding for international characters
\usepackage{amsmath}
\usepackage{amsfonts}
\usepackage{pifont}
\usepackage[backend=biber,
style=apa,
citestyle=authoryear]{biblatex}

\addbibresource{references.bib}

\usepackage{mathpazo} % Palatino font

\begin{document}

%----------------------------------------------------------------------------------------
%	TITLE PAGE
%----------------------------------------------------------------------------------------

\begin{titlepage} % Suppresses displaying the page number on the title page and the subsequent page counts as page 1
	\newcommand{\HRule}{\rule{\linewidth}{0.5mm}} % Defines a new command for horizontal lines, change thickness here

	\center % Centre everything on the page

	%------------------------------------------------
	%	Headings
	%------------------------------------------------

	\textsc{\LARGE American Public University}\\[1.5cm] % Main heading such as the name of your university/college

	%------------------------------------------------
	%	Title
	%------------------------------------------------

	\HRule\\[0.4cm]

	{\huge\bfseries Assignment 4}\\[0.4cm] % Title of your document

	\HRule\\[1.5cm]

	%------------------------------------------------
	%	Author(s)
	%------------------------------------------------

	\begin{minipage}{0.4\textwidth}
		\begin{flushleft}
			\large
			\textit{Author}\\
			Daniel \textsc{Justice} % Your name
		\end{flushleft}
	\end{minipage}
	~
	\begin{minipage}{0.4\textwidth}
		\begin{flushright}
			\large
			\textit{Professor}\\
			Dr. Farshad \textsc{Foroozan} % Supervisor's name
		\end{flushright}
	\end{minipage}

	%------------------------------------------------
	%	Date
	%------------------------------------------------

	\vfill\vfill\vfill
	{\large\today} % Date, change the \today to a set date if you want to be precise
	\vfill

\end{titlepage}

%----------------------------------------------------------------------------------------

\section{Prove the following by using: a direct argument; definition 4.5; and definition 4.7.}

$x_0=1$ is both an accumulation point and a boundary point of the interval $(1, 2) \in \mathbb{R}$.

\paragraph{}

By Definition 4.5, let $E$ be a set of real numbers. Any point $x$ (not necessarily in $E$) is said to be an accumulation point of $E$ provided that for every $c>0$ the intersection $(x-c, x+x) \cap E$ contains infinitely many points.
$1$ is not in the set $(1, 2)$.
For every $c>0$, $x+c$ contains infinitely many points.
For every natural number $n$, $$1 < \frac{1}{c+n} < 1+c$$
There are infinitely many naturals, and each of these lies on the interval $(1, 1+c)$.

\paragraph{}

By Definition 4.7, Let $E$ be a set of real numbers. Any point $x$ (not ncessarily in $E$) is said to be a boundary point of $E$ provided that every interval $(x-x, x+c)$ contains at least one point of $E$ and also at least one point that does not belong to $E$.
For ever $c>0$, $1+c$ lies inside the interval $(1, 2)$ because $1<1+c$.
For ever $c>0$, $1-c$ lies outside the interval $(1, 2)$ because $1-c \le 1$.

$\square$

\section{Prove the following by using: a direct argument and definition 5.1.}

Prove this limit for $$f(x)=\sqrt{x}: \lim _{x\to 4} f(x)=2$$.

\paragraph{}

By Definition 5.1, Let $f:E \to \mathbb{R}$ be a function with domain $E$ and suppose that $x_0$ is a point of accumulation of $E$.
Then we write $$lim_{x \to x_0} f(x) = L$$
if for every $\epsilon > 0$ there is a $\delta > 0$ so that $$|f(x)-L|< \epsilon$$
whenever $x$ is a point of $E$ differing from $x_0$ and satisfying $|x-x_0|<\delta$.

\begin{equation}
	\begin{aligned}
		f(x) = \sqrt{x} \\
		f(4) = 2 \\
		|\sqrt{x} - 2| < \epsilon \\
		|\sqrt{x}| - |2| \le |\sqrt{x} - 2| < \epsilon \\
		|\sqrt{x}| < \epsilon + 2 \\
	\end{aligned}
\end{equation}

Let $\epsilon > 0$.
Let $\delta = \epsilon + 2$.
Then for all $x$ with $|\sqrt{x}| < \delta$, we have $$|\sqrt{x}-4| < \delta < \epsilon + 2$$.
By definition, $lim_{x \to 4} = 2$ as required.

$\square$

\section{Prove the following by using: a direct argument; definition 4.12; and Theorem 5.37.}

The function $f(x)=e^{x/2} $ is continuous at $x=3$.

\paragraph{}

By Theorem 5.37, Let $f:A \to \mathbb{R}$.
Then $f$ is continuous if and only if for every open set $V \subset \mathbb{R}$, the set $f^{-1}(V) = \{x \in A: f(x) \in V \}$.

By Definition 4.12, Let $E$ be a set of real numbers.
Then $E$ is said to be open if every point of $E$ is also an interior point of $E$.

Let $f(x) = e^{x/2}$.
Choose $\alpha < f(x) < \beta$.
We find
\begin{equation}
	\begin{aligned}
		f^{-1}((\alpha, \beta)) = \big ( e^{\alpha / 2}, e^{\beta / 2} \big ) \\
	\end{aligned}
\end{equation}

Since $\big ( e^{\alpha / 2}, e^{\beta / 2} \big )$ is open it would follow that $f$ is continuous at $x=3$.

$\square$

\printbibliography

\end{document}
