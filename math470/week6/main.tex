\documentclass{article}
\usepackage{amsmath}
\usepackage{amsfonts}
\usepackage{tabularx}

\begin{document}

\section*{5.1}

\begin{tabularx}{\textwidth}{ |X|c|c| }
    \hline
	{\bf Definition} & {\bf Algorithm} & {\bf Questions} \\
    \hline
	Definition 5.1, Lipschitz condition & & \\
    \hline
	Definition 5.2, Convex sets & & \\
    \hline
	Theorem 5.3, function satisfying Lipschitz condition & & \\
    \hline
	Theorem 5.4, function satisfying Lipschitz condition has unique solution & & \\
    \hline
	Definition 5.5, Well-posed IVP & & \\
    \hline
	Theorem 5.6, function satisfying Lipschitz condition is well-posed & & \\
    \hline
\end{tabularx}

\section*{5.2}

\begin{tabularx}{\textwidth}{ |X|X|c| }
    \hline
	{\bf Definition} & {\bf Algorithm} & {\bf Questions} \\
    \hline
	 &  Algorithm 5.1, Euler's method & \\
    \hline
	Lemma 5.7, Bounds of Euler's method & & \\
    \hline
	Lemma 5.8, sequence of Euler's method converges & & \\
    \hline
	Theorem 5.9, $f$ satisfying Lipschitz and continuous is approximated by Euler's & & \\
    \hline
	Theorem 5.10, Error bounds of Euler's & & \\
    \hline
\end{tabularx}

\section*{5.3}

\begin{tabularx}{\textwidth}{ |X|c|c| }
    \hline
	{\bf Definition} & {\bf Algorithm} & {\bf Questions} \\
    \hline
	Definition 5.11, local truncation error & & \\
    \hline
	Theorem 5.12, local truncation error of higher-order Taylor & & \\
    \hline
\end{tabularx}

\section*{5.4}

\begin{tabularx}{\textwidth}{ |X|X|c| }
    \hline
	{\bf Definition} & {\bf Algorithm} & {\bf Questions} \\
    \hline
	Theorem 5.13, $n$th Taylor polynomial in two variables & & \\
    \hline
	 & Algorithm 5.2, Order Four Runge-Kutta & \\
    \hline
\end{tabularx}

\section*{5.5}

\begin{tabularx}{\textwidth}{ |X|X|c| }
    \hline
	{\bf Definition} & {\bf Algorithm} & {\bf Questions} \\
    \hline
	 & Algorithm 5.3, Runge-Kutta-Fehlberg &\\
    \hline
\end{tabularx}

\section*{Remarks}

TBD

\end{document}
