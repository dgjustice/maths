\documentclass{article}
\usepackage{amsmath}
\usepackage{amsfonts}
\usepackage{tabularx}

\begin{document}

\section*{3.1}

\begin{tabularx}{\textwidth}{ |X|X|X| }
    \hline
	{\bf Definition} & {\bf Algorithm} & {\bf Questions} \\
    \hline
	 & Algorithm 2.1, Bisection algorithm & none \\
    \hline
	Theorem 3.1, Weierstrass Approximation Theorem, for each $\epsilon > 0$, there exists a polynomial $P(x)$, with the property that $|f(x)-P(x)|<\epsilon$. & & none \\
    \hline
	Theorem 3.2, Lagrange interpolating polynomial. $$\sum_{k=0}^{n}f(x_k)L_{n,k}(x)$$ & & none \\
    \hline
	Theorem 3.3, Lagrange interpolating polynomial error term. $$P(x)+\frac{f^{(n+1)}(\xi(x))}{(n+1)!}$$ $$(x-x_0)(x-x_1)...(x-x_n)$$ & & none \\
    \hline
\end{tabularx}

\section*{3.2}

\begin{tabularx}{\textwidth}{ |X|X|X| }
    \hline
	{\bf Definition} & {\bf Algorithm} & {\bf Questions} \\
    \hline
	Definition 3.4, Neville's Method & & none \\
    \hline
	Theorem 3.5, Neville's Method, Lagrange polynomial at $P(x)$  & & none \\
    \hline
	 & Algorithm 3.1, Neville's Iterated Interpolation & Straightforward to implement \\
    \hline
\end{tabularx}

\section*{3.3}

\begin{tabularx}{\textwidth}{ |X|X|X| }
    \hline
	{\bf Definition} & {\bf Algorithm} & {\bf Questions} \\
    \hline
	 & Algorithm 3.2, Newton's Divided-Difference Formula & This was a challenge to get right. \\
    \hline
	Theorem 3.6, Newton's DDF error term & & none \\
    \hline
	Definition 3.7, Newton's backward-difference formula & & This was challenging as well. \\
    \hline
\end{tabularx}

\section*{3.4}

\begin{tabularx}{\textwidth}{ |X|X|X| }
    \hline
	{\bf Definition} & {\bf Algorithm} & {\bf Questions} \\
    \hline
	Definition 3.8, osculating polynomial & & none \\
    \hline
	Theorem 3.9, Hermite polynomial & & none \\
    \hline
	 & Algorithm 3.3, Hermite Interpolation & The linear algebra approach is far simpler. \\
    \hline
\end{tabularx}

\section*{Remarks}

When I graphed my interpolated function on top of the real function for exercise 3.1-8b, I was shocked by how close it is on the interval, but how far it diverges beyond that.

The algorithms and techniques in section 3.3 were very challenging to get correct, and the various pieces were hard to put together.

Section 3.4 was as challenging as the other sections.
The linear algebra approach to solving Hermite interpolation is far simpler than the method in the book.
This is something I would like to revisit in the future.

\end{document}
