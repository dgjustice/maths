\documentclass{article}
\usepackage{amsmath}
\usepackage{amsfonts}
\usepackage{pifont}

\begin{document}

\section{1.3.6}

Define operations of addition and multiplication on $\mathbb{Z}_5 = \{0,1,2,3,4\}$ as follows:

\begin{tabular}{|c|c|c|c|c|c||c|c|c|c|c|c|}
    \hline
    + & 0 & 1 & 2 & 3 & 4 & x & 0 & 1 & 2 & 3 & 4 \\
    \hline
    0 & 0 & 1 & 2 & 3 & 4 & 0 & 0 & 0 & 0 & 0 & 0 \\
    \hline
    1 & 1 & 2 & 3 & 4 & 0 & 1 & 0 & 1 & 2 & 3 & 4 \\
    \hline
    2 & 2 & 3 & 4 & 0 & 1 & 2 & 0 & 2 & 4 & 1 & 3 \\
    \hline
    3 & 3 & 4 & 0 & 1 & 2 & 3 & 0 & 3 & 1 & 4 & 2 \\
    \hline
    4 & 4 & 0 & 1 & 2 & 3 & 4 & 0 & 4 & 3 & 2 & 1 \\
    \hline
\end{tabular}

\paragraph{}

Show that $\mathbb{Z}_5$ satisfies all the field axioms.

\subsection{A1}

For any $a, b\in\mathbb{Z}_5$ there is a number $a+b\in\mathbb{Z}_5$ and $a + b = b + a$.

\begin{itemize}
    \item $a+b\in\mathbb{Z}_5$ can be determined by the given definition of addition.  No element of $\mathbb{Z}_5$ is $>4$, so the operation of addition is closed.
    \item This follows directly from examining the addition table.
\end{itemize}

$a + b = b + a$ can be validated by examining the table and enumerating the combinations of rows and columns.  I have provided an example for $0$ below.  The remaining integers follow the same pattern.
\begin{itemize}
	\item $0+0=0$
	\item $0+1=1+0=1$
	\item $0+2=2+0=2$
	\item $0+3=3+0=3$
	\item $0+4=4+0=4$
	\item $1+1=1$
	\item $1+2=2+1=3$
	\item ...
\end{itemize}

\subsection{A2}

For any $a, b, c\in\mathbb{Z}_5$ the indentity $(a+b)+c=a+(b+c)$ is true.

\begin{itemize}
    \item Since addition in $\mathbb{Z}_5$ is closed, every sum $a+b$ and $b+c$ falls inside the set.  This can be verified computationally in the case of $\mathbb{Z}_5$, but a more useful proof would require better definitions of modular arithmetic and a succcessor function.
\end{itemize}

\subsection{A3}

There is a unique number $0\in\mathbb{Z}_5$ so that, for all $a\in\mathbb{Z}_5$, $a+0=0+a=a$.

\begin{itemize}
    \item This follows directly from examining the addition table.
    \begin{itemize}
        \item[\ding{217}] $0+0=0$
        \item[\ding{217}] $0+1=1+0=1$
        \item[\ding{217}] $0+2=2+0=2$
        \item[\ding{217}] $0+3=3+0=3$
        \item[\ding{217}] $0+4=4+0=4$
    \end{itemize}
\end{itemize}


\subsection{A4}

For any number $a\in\mathbb{Z}_5$ there is a corresponding number denoted by $-a$ with the property that $a+(-a)=0$.

\begin{itemize}
	\item $0+(-0)=0$
	\item $1+(-1)=0$
	\item $2+(-2)=0$
	\item $3+(-3)=0$
	\item $4+(-4)=0$
\end{itemize}

\subsection{M1}

For any $a, b\in\mathbb{Z}_5$ there is a number $ab\in\mathbb{Z}_5$ and $ab=ba$.

\begin{itemize}
    \item $a*b\in\mathbb{Z}_5$ can be determined by the given definition of multiplication.  No element of $\mathbb{Z}_5$ is $>4$, so the operation of multiplication is closed.  Similar to the case for addition, this can be validated computationally (with a computer).
\end{itemize}

\subsection{M2}

For any $a, b, c\in\mathbb{Z}_5$ the identity $(ab)c=a(bc)$ is true.

\begin{itemize}
	\item The case of any $a, b, c$ equal to one is covered by M3 and M4.  Any $a, b, c$ equal to zero results in 0.  The remaining cases can be validated computationally.
\end{itemize}

\subsection{M3}

There is a unique number $1\in\mathbb{Z}_5$ so that $a1=1a=a$ for all $a\in\mathbb{Z}_5$.

\begin{itemize}
    \item This follows directly from examining the multiplication table.
    \begin{itemize}
        \item[\ding{217}] $1*1=1$
        \item[\ding{217}] $1*1=1*1=1$
        \item[\ding{217}] $1*2=2*1=2$
        \item[\ding{217}] $1*3=3*1=3$
        \item[\ding{217}] $1*4=4*1=4$
    \end{itemize}
\end{itemize}

\subsection{M4}

For any number $a\in\mathbb{Z}_5, a\neq0$, there is a corresponding number denoted $a^{-1}$ with the property that $aa^{-1}=1$.

\begin{itemize}
    \item This follows directly from examining the multiplication table.
    \begin{itemize}
        \item[\ding{217}] $1*1=1$
        \item[\ding{217}] $2*3=1$
        \item[\ding{217}] $3*2=1$
        \item[\ding{217}] $4*4=1$
    \end{itemize}
\end{itemize}

\subsection{AM1}

For any $a, b, c \in\mathbb{Z}_5$ the identity $(a+b)c=ac +bc$ is true.

\begin{itemize}
	\item The case of $c=0$ is trivially 0. $c=1$ is covered by M3.  The remaining cases can be validated computationally.
\end{itemize}

\end{document}
