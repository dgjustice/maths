\documentclass[11pt]{article}
\usepackage[utf8]{inputenc} % Required for inputting international characters
\usepackage[T1]{fontenc} % Output font encoding for international characters
\usepackage{amsmath}
\usepackage{amsfonts}
\usepackage{pifont}
\usepackage[backend=biber,
style=apa,
citestyle=authoryear]{biblatex}

\addbibresource{references.bib}

\usepackage{mathpazo} % Palatino font

\begin{document}

%----------------------------------------------------------------------------------------
%	TITLE PAGE
%----------------------------------------------------------------------------------------

\begin{titlepage} % Suppresses displaying the page number on the title page and the subsequent page counts as page 1
	\newcommand{\HRule}{\rule{\linewidth}{0.5mm}} % Defines a new command for horizontal lines, change thickness here

	\center % Centre everything on the page

	%------------------------------------------------
	%	Headings
	%------------------------------------------------

	\textsc{\LARGE American Public University}\\[1.5cm] % Main heading such as the name of your university/college

	%------------------------------------------------
	%	Title
	%------------------------------------------------

	\HRule\\[0.4cm]

	{\huge\bfseries Assignment 2}\\[0.4cm] % Title of your document

	\HRule\\[1.5cm]

	%------------------------------------------------
	%	Author(s)
	%------------------------------------------------

	\begin{minipage}{0.4\textwidth}
		\begin{flushleft}
			\large
			\textit{Author}\\
			Daniel \textsc{Justice} % Your name
		\end{flushleft}
	\end{minipage}
	~
	\begin{minipage}{0.4\textwidth}
		\begin{flushright}
			\large
			\textit{Professor}\\
			Dr. Farshad \textsc{Foroozan} % Supervisor's name
		\end{flushright}
	\end{minipage}

	%------------------------------------------------
	%	Date
	%------------------------------------------------

	\vfill\vfill\vfill
	{\large\today} % Date, change the \today to a set date if you want to be precise
	\vfill

\end{titlepage}

%----------------------------------------------------------------------------------------

\section{}

\#1) Prove the following by using: a direct argument and definition 2.6.
\begin{equation}
	\lim_{n \rightarrow \infty} \bigg(\frac{3n^2}{2n^2-3n}\bigg)=\frac{3}{2}\text{ for }n \in \mathbb{N}
\end{equation}

\paragraph{Value of the limit}
\begin{equation}
	\begin{aligned}
		= \lim_{n \rightarrow \infty} \bigg(\frac{3n^2}{2n^2-3n}\bigg) \\
		= \lim_{n \rightarrow \infty} \bigg(\frac{3n^2}{2n^2-3n}\bigg) \cdot \frac{\frac{1}{n^2}}{\frac{1}{n^2}} \\
		= \lim_{n \rightarrow \infty} \bigg(\frac{3}{2-\frac{3}{n}}\bigg) \\
		= \frac{\displaystyle\lim_{n \rightarrow \infty} 3}{\displaystyle\lim_{n \rightarrow \infty} 2 - \displaystyle\lim_{n \rightarrow \infty} \frac{3}{n}} \\
		= \frac{\displaystyle\lim_{n \rightarrow \infty} 3}{\displaystyle\lim_{n \rightarrow \infty} 2 - 0} \\
		= \frac{\displaystyle\lim_{n \rightarrow \infty} 3}{\displaystyle\lim_{n \rightarrow \infty} 2} \\
		= \frac{3}{2} \\
	\end{aligned}
\end{equation}

\paragraph{Proof of the limit}

Let any positive $\epsilon$ be given.  We need to find a number N so that every term in the sequence on and after the Nth term is closer to 3/2 than $\epsilon$, that is, so that
\begin{equation}
	\bigg|\bigg(\frac{3n^2}{2n^2-3n}\bigg) - \frac{3}{2}\bigg| < \epsilon
\end{equation}
for n=N, n=N+1, n=N+2, ... (\cite{tbb})

\begin{equation}
	\begin{aligned}
		\bigg|\bigg(\frac{3n^2}{2n^2-3n}\bigg) - \frac{3(2n^2-3n)}{2(2n^2-3n)}\bigg| < \epsilon \\
		\bigg|\bigg(\frac{3n^2}{2n^2-3n}\bigg) - \frac{6n^2-9n}{2(2n^2-3n))}\bigg| < \epsilon \\
		\bigg|\frac{-3n^2+9n}{-2n^2+3n}\bigg| < \epsilon \\
		\frac{3n^2-9n}{2n^2-3n} > -\epsilon \\
	\end{aligned}
\end{equation}

$\square$

\section{}

\#2) Prove the following by using: a direct argument; exercise 2.4.1; and Theorems 2.14, 2.15, 2.16, and/or 2.17.
\begin{equation}
	\lim_{n \rightarrow \infty} \big(\sqrt{n^2+n}-n\big)=\frac{1}{2}\text{ for }n \in \mathbb{N}
\end{equation}

\begin{equation}
	\begin{aligned}
		\lim_{n \rightarrow \infty} \big(\sqrt{n^2+n}-n\big) \bigg(\frac{\sqrt{n^2+n}+n}{\sqrt{n^2+n}+n} \bigg) \\
		\lim_{n \rightarrow \infty} \frac{n}{\sqrt{n^2+n}+n} \\
		\lim_{n \rightarrow \infty} \frac{n}{\sqrt{n^2+\frac{n^2}{n}}+n} \\
		\lim_{n \rightarrow \infty} \frac{n}{\sqrt{n^2(1+\frac{1}{n})}+n} \\
		\lim_{n \rightarrow \infty} \frac{n}{\sqrt{n^2}\sqrt{(1+\frac{1}{n})}+n} \\
		\lim_{n \rightarrow \infty} \frac{n}{n\sqrt{(1+\frac{1}{n})}+n} \\
		\frac{\displaystyle\lim_{n \rightarrow \infty}n}{\displaystyle\lim_{n \rightarrow \infty}n\cdot\displaystyle\lim_{n \rightarrow \infty}\sqrt{(1+\frac{1}{n})}+\displaystyle\lim_{n \rightarrow \infty}n} \\
		\frac{\displaystyle\lim_{n \rightarrow \infty}n}{\displaystyle\lim_{n \rightarrow \infty}n\cdot1+\displaystyle\lim_{n \rightarrow \infty}n} \\
		\frac{\displaystyle\lim_{n \rightarrow \infty}n}{\displaystyle\lim_{n \rightarrow \infty}n+\displaystyle\lim_{n \rightarrow \infty}n} \\
		\frac{\displaystyle\lim_{n \rightarrow \infty}n}{\displaystyle\lim_{n \rightarrow \infty}n+\displaystyle\lim_{n \rightarrow \infty}n} \\
		\frac{\displaystyle\lim_{n \rightarrow \infty}n}{2\cdot\displaystyle\lim_{n \rightarrow \infty}n} \\
		=\frac{1}{2}
	\end{aligned}
\end{equation}

$\square$

\section{}

\#3) Prove the following by using: a direct argument; and definition 2.9.
\begin{equation}
	\lim_{n \rightarrow \infty} \bigg(\frac{e^{2n}}{e^n+1}\bigg)\text{ diverges for }n \in \mathbb{N}
\end{equation}

If M is any positive number we need to find some point in the sequence after which all terms exceed M. Thus we need to consider the inequality (\cite{tbb})

\begin{equation}
	\begin{aligned}
		\frac{e^{2n}}{e^n+1}\ge M \\
		\frac{e^2 e^n}{e^n+1}\ge M \\
	\end{aligned}
\end{equation}

Since

\begin{equation}
	\frac{e^{n}}{e^n+1} < 1
\end{equation}

as long as n $\ge$M+1 this will be true. Thus take any integer N $\ge$ M + 1 and it will be true that

\begin{equation}
	\frac{e^{2n}}{e^n+1}\ge M
\end{equation}

for all n $\ge$ N.

$\square$

\printbibliography

\end{document}
