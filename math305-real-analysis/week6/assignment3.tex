\documentclass[11pt]{article}
\usepackage[utf8]{inputenc} % Required for inputting international characters
\usepackage[T1]{fontenc} % Output font encoding for international characters
\usepackage{amsmath}
\usepackage{amsfonts}
\usepackage{pifont}
\usepackage[backend=biber,
style=apa,
citestyle=authoryear]{biblatex}

\addbibresource{references.bib}

\usepackage{mathpazo} % Palatino font

\begin{document}

%----------------------------------------------------------------------------------------
%	TITLE PAGE
%----------------------------------------------------------------------------------------

\begin{titlepage} % Suppresses displaying the page number on the title page and the subsequent page counts as page 1
	\newcommand{\HRule}{\rule{\linewidth}{0.5mm}} % Defines a new command for horizontal lines, change thickness here

	\center % Centre everything on the page

	%------------------------------------------------
	%	Headings
	%------------------------------------------------

	\textsc{\LARGE American Public University}\\[1.5cm] % Main heading such as the name of your university/college

	%------------------------------------------------
	%	Title
	%------------------------------------------------

	\HRule\\[0.4cm]

	{\huge\bfseries Writing Assignment 3}\\[0.4cm] % Title of your document

	\HRule\\[1.5cm]

	%------------------------------------------------
	%	Author(s)
	%------------------------------------------------

	\begin{minipage}{0.4\textwidth}
		\begin{flushleft}
			\large
			\textit{Author}\\
			Daniel \textsc{Justice} % Your name
		\end{flushleft}
	\end{minipage}
	~
	\begin{minipage}{0.4\textwidth}
		\begin{flushright}
			\large
			\textit{Professor}\\
			Dr. Farshad \textsc{Foroozan} % Supervisor's name
		\end{flushright}
	\end{minipage}

	%------------------------------------------------
	%	Date
	%------------------------------------------------

	\vfill\vfill\vfill
	{\large\today} % Date, change the \today to a set date if you want to be precise
	\vfill

\end{titlepage}

%----------------------------------------------------------------------------------------

\section{}

\#1) Prove the following by using mathematical induction.

\paragraph{}

For all n $\in \mathbb{N}$, $5^{2n}-1$ is a multiple of 8.

\paragraph{}

Rewrite $5^{2(n+1)}-1$ as $(8\cdot3 + 1)^n-1$.
\paragraph{}
Base case:

\begin{equation}
	\begin{aligned}
		(8\cdot3+1)^1-1=3\cdot8 \\
		8\cdot3+1-1=3\cdot8 \\
		8\cdot3=3\cdot8 \\
	\end{aligned}
\end{equation}

Assume $5^{2n}-1$ is true.  By the binomial theorem:

\begin{equation}
	\begin{aligned}
		(8\cdot3+1)^n-1=\bigg[\sum_{k=0}^{n} \binom{n}{k}(8\cdot3)^{n-k}1^k\bigg] - 1 \\
		(8\cdot3+1)^n-1=\bigg[\sum_{k=0}^{n} \binom{n}{k}(8\cdot3)^{n-k}\bigg] - 1 \\
	\end{aligned}
\end{equation}

Then by induction:

\begin{equation}
	\begin{gathered}
		(8\cdot3+1)^{n+1}-1=\bigg[\sum_{k=0}^{n+1} \binom{n+1}{k}(8\cdot3)^{n+1-k}1^k\bigg] - 1 \\
		(8\cdot3+1)^{n}\bigg[\sum_{k=0}^{n} \binom{n}{k}(8\cdot3)^{n-k}\bigg] - 1 =\bigg[\sum_{k=0}^{n+1} \binom{n+1}{k}(8\cdot3)^{n+1-k}\bigg] - 1 \\
		(8\cdot3+1)^{n}\bigg[\sum_{k=0}^{n} \binom{n}{k}(8\cdot3)^{n-k}\bigg] =\bigg[\sum_{k=0}^{n+1} \binom{n+1}{k}(8\cdot3)^{n+1-k}\bigg] \\
		(8\cdot3+1)^{n}\bigg[\sum_{k=0}^{n} \binom{n}{k}(8\cdot3)^{n-k}\bigg] =(8\cdot3+1)^{n}\bigg[\sum_{k=0}^{n} \binom{n}{k}(8\cdot3)^{n-k}\bigg] \\
		(8\cdot3+1)^{n} =(8\cdot3+1)^{n} \\
	\end{gathered}
\end{equation}

$\square$

\section{}

\#2) Prove the following by using mathematical induction.

\paragraph{}

For all n $\in \mathbb{N}$ with an index i $\in\mathbb{N}$;

\begin{equation}
	\sum_{i=1}^{n}i^2=\frac{n(n+1)(2n+1)}{6}
\end{equation}

Base case:
\begin{equation}
	\begin{gathered}
		\sum_{i=1}^{1}i^2=\frac{1(1+1)(2\cdot1+1)}{6} \\
		1^2=\frac{(2)(3)}{6} \\
		1 = \frac{6}{6} \\
		1 = 1 \\
	\end{gathered}
\end{equation}

Assume the following is true:

\begin{equation}
	1^2+2^2+3^2...+{n}^2=\frac{n(n+1)(2n+1)}{6}
\end{equation}

We wish to show that
\begin{equation}
	1^2+2^2+3^2...+{n}^2+(n+1)^2=\frac{(n(n+1)(2n+1)}{6}
\end{equation}

also holds true. By induction:

\begin{equation}
	\begin{gathered}
		1^2+2^2+3^2+...n^2+(n+1)^2=\frac{(n+1)((n+1)+1)(2(n+1)+1)}{6} \\
		1^2+2^2+3^2+...n^2+(n+1)^2=\frac{(n+1)(n+2)(2n+3)}{6} \\
		1^2+2^2+3^2+...n^2+(n+1)^2= \\
		\frac{(n+1)(n+2)(2n+3)}{6} = \frac{1(1+1)(2\cdot1+1)}{6} + (1+n)^2 \\
		=\frac{2n^3+9n^2+13n+6}{6} = \frac{2n^3+2n^2+n^2+n}{6} + \frac{6}{6}(n^2+2n+1) \\
		=\frac{2n^3+9n^2+13n+6}{6} = \frac{2n^3+2n^2+n^2+n}{6} + \frac{6n^2+12n+6)}{6} \\
		=\frac{2n^3+9n^2+13n+6}{6} = \frac{2n^3+9n^2+13n+6}{6} \\
	\end{gathered}
\end{equation}

$\square$

\#3) Prove the following by using: a direct argument; definition 3.6; and Theorem 2.28.

\paragraph{}

Prove the following series diverges:

\begin{equation}
	\sum_{k=1}^{\infty}\frac{k}{5k+1}
\end{equation}

Definition 3.6 states that a series converges if the sequence of partial sums converges.
If the sequence of partial sums does not converge, then the series is divergent (\cite{tbb}).

Theorem 2.28 states if we are given a monotonic sequence ${s_n}$, then it is convergent if and only if ${s_n}$ is bounded (\cite{tbb}).

The sequence $\frac{k}{5k+1}$ is bounded with a limit approaching $\frac{1}{5}$, but the series is divergent.

The smallest term in the sequence of partial sums is $\frac{1}{6}$.
Choose any N $\ge$ 6M, and:

\begin{equation}
	\sum_{k=1}^{N}\frac{k}{5k+1} \ge M
\end{equation}

thereby showing directly that the series diverges.

$\square$

\printbibliography

\end{document}
